%% start of file `template.tex'.
%% Copyright 2006-2015 Xavier Danaux (xdanaux@gmail.com).
%
% This work may be distributed and/or modified under the
% conditions of the LaTeX Project Public License version 1.3c,
% available at http://www.latex-project.org/lppl/.


\documentclass[11pt,a4paper,sans]{moderncv}        % possible options include font size ('10pt', '11pt' and '12pt'), paper size ('a4paper', 'letterpaper', 'a5paper', 'legalpaper', 'executivepaper' and 'landscape') and font family ('sans' and 'roman')

% moderncv themes
\moderncvstyle{casual}                             % style options are 'casual' (default), 'classic', 'banking', 'oldstyle' and 'fancy'
\moderncvcolor{blue}                               % color options 'black', 'blue' (default), 'burgundy', 'green', 'grey', 'orange', 'purple' and 'red'
%\renewcommand{\familydefault}{\sfdefault}         % to set the default font; use '\sfdefault' for the default sans serif font, '\rmdefault' for the default roman one, or any tex font name
%\nopagenumbers{}                                  % uncomment to suppress automatic page numbering for CVs longer than one page

% character encoding
\usepackage[utf8]{inputenc}                       % if you are not using xelatex ou lualatex, replace by the encoding you are using
%\usepackage{CJKutf8}                              % if you need to use CJK to typeset your resume in Chinese, Japanese or Korean

% adjust the page margins
\usepackage[scale=0.75]{geometry}
%\setlength{\hintscolumnwidth}{3cm}                % if you want to change the width of the column with the dates
%\setlength{\makecvtitlenamewidth}{10cm}           % for the 'classic' style, if you want to force the width allocated to your name and avoid line breaks. be careful though, the length is normally calculated to avoid any overlap with your personal info; use this at your own typographical risks...

% personal data
\name{Francesco}{Mattiotti}
\title{Curriculum vitae et studiorum}              % optional, remove / comment the line if not wanted
\address{Via Prezzolaro, 18}{25080 Manerba del Garda (BS)}{Italy}% optional, remove / comment the line if not wanted; the "postcode city" and "country" arguments can be omitted or provided empty
\phone[mobile]{+39~334~9342183}                   % optional, remove / comment the line if not wanted; the optional "type" of the phone can be "mobile" (default), "fixed" or "fax"
% \phone[fixed]{+39~0365~552137}
%\phone[fax]{+3~(456)~789~012}
\email{mattiottifrancesco@gmail.com}                               % optional, remove / comment the line if not wanted
% \homepage{dmf.unicatt.it/~fmatt91}                         % optional, remove / comment the line if not wanted
% \social[linkedin]{francesco-mattiotti-a76813144}                        % optional, remove / comment the line if not wanted
%\social[twitter]{jdoe}                             % optional, remove / comment the line if not wanted
% \social[github]{matiot}                              % optional, remove / comment the line if not wanted
%\extrainfo{additional information}                 % optional, remove / comment the line if not wanted
% \photo[64pt][0.4pt]{mugshot.jpg}                       % optional, remove / comment the line if not wanted; '64pt' is the height the picture must be resized to, 0.4pt is the thickness of the frame around it (put it to 0pt for no frame) and 'picture' is the name of the picture file
%\quote{Some quote}                                 % optional, remove / comment the line if not wanted

% bibliography adjustements (only useful if you make citations in your resume, or print a list of publications using BibTeX)
%   to show numerical labels in the bibliography (default is to show no labels)
%\makeatletter\renewcommand*{\bibliographyitemlabel}{\@biblabel{\arabic{enumiv}}}\makeatother
%   to redefine the bibliography heading string ("Publications")
\renewcommand{\refname}{Publications and preprints}
\usepackage{etaremune}
\makeatletter
\long\def\thebibliography#1{%
  \section*{\refname}%
  \@mkboth{\MakeUppercase\refname}{\MakeUppercase\refname}
  \settowidth{\dimen0}{\@biblabel{#1}}%
  \setlength{\dimen2}{\dimen0}%
  \addtolength{\dimen2}{\labelsep}
  \sloppy
  \clubpenalty 4000 
  \@clubpenalty 
  \clubpenalty 
  \widowpenalty 4000
  \sfcode `\.\@m
  \renewcommand{\labelenumi}{\@biblabel{\theenumi}} % labels like [3], [2], [1]
  \begin{etaremune}[labelwidth=\dimen0,leftmargin=\dimen2]\@openbib@code
}
\def\endthebibliography{\end{etaremune}}
\def\bibitem#1{%
 \item \if@filesw\immediate\write\@auxout{\string\bibcite{#1}{\the\value{enumi}}}\fi\ignorespaces
}
\makeatother
% bibliography with mutiple entries
%\usepackage{multibib}
%\newcites{book,misc}{{Books},{Others}}
%----------------------------------------------------------------------------------
%            content
%----------------------------------------------------------------------------------
\begin{document}
%\begin{CJK*}{UTF8}{gbsn}                          % to typeset your resume in Chinese using CJK
%-----       resume       ---------------------------------------------------------
\makecvtitle

\section{Education}
\cventry{2016}{Master's degree in Physics}{cum laude}{February 16\textsuperscript{th}, 2016, at Facoltà di Scienze Matematiche, Fisiche e Naturali - Università Cattolica del Sacro Cuore (Brescia, Italy)}{}{}  % arguments 3 to 6 can be left empty
\cvitem{}{With a thesis on the interplay of cooperativity and noise, from light-harvesting complexes to quantum transport. Advisor: Giuseppe Luca Celardo. Co-advisor: Fausto Borgonovi.}
\cventry{2013}{Bachelor's degree in Physics}{cum laude}{December 16\textsuperscript{th}, 2013, at Facoltà di Scienze Matematiche, Fisiche e Naturali - Università Cattolica del Sacro Cuore (Brescia, Italy)}{}{}
\cvitem{}{With a thesis on the non-Hermitian Hamiltonian approach to quantum transport. Advisor: Giuseppe Luca Celardo. Co-advisor: Giulio Giuseppe Giusteri.}
\cventry{2010}{High school diploma}{(classical studies)}{at Liceo ``G.~Bagatta'', Desenzano del Garda (Brescia, Italy)}{}{}

%\cvitem{description}{Short thesis abstract}

\section{Employments}
%\subsection{Vocational}
%{\newline{}%
% Detailed achievements:%
% \begin{itemize}%
% \item Achievement 1;
% \item Achievement 2, with sub-achievements:
%   \begin{itemize}%
%   \item Sub-achievement (a);
%   \item Sub-achievement (b), with sub-sub-achievements (don't do this!);
%     \begin{itemize}
%     \item Sub-sub-achievement i;
%     \item Sub-sub-achievement ii;
%     \item Sub-sub-achievement iii;
%     \end{itemize}
%   \item Sub-achievement (c);
%   \end{itemize}
% \item Achievement 3.
% \end{itemize}}
% \subsection{Miscellaneous}
% \cventry{year--year}{Job title}{Employer}{City}{}{Description}
\cventry{2017-now}{PhD Student with scholarship}{International PhD in Science}{on a joint research project between Università Cattolica del Sacro Cuore and University of Notre Dame du Lac}{}{}
\cvitem{}{The research project is about Cooperative Effects in quantum systems, supervised by Prof. Fausto Borgonovi, Prof. Giuseppe Luca Celardo and Prof. Boldizs\'ar Jank\'o.}
\cventry{2017}{Teaching Assistant}{Quantum Mechanics}{20 hours of exercises at Università Cattolica del Sacro Cuore (Brescia, Italy)}{}{}
\cventry{2016}{Research Assistant}{}{at Facoltà di Scienze Matematiche, Fisiche e Naturali - Università Cattolica del Sacro Cuore (Brescia, Italy)}{}{}
\cvitem{}{On a project about quantum transport in nanostructured systems with applications to biosystems, coordinated by Prof. Fausto Borgonovi. The project is financed by Fondazione EULO.}

% \newpage


\section{Transferable skills}

\cvitem{Group work and tutoring}{I had international collaborations with experienced researchers and with other PhD students. I also trained and tutored undergraduate and graduate students.}
\cvitem{Communication skills}{I have given various oral presentations to international conferences/gatherings. I have written papers that were published on peer-reviewed journals.}


\section{Language skills}
\cvitemwithcomment{Italian}{native speaker.}{}
\cvitemwithcomment{English}{professional working proficiency;}{IELTS - Academic score: 7.0/9 (CEFR level: C1).}
\cvitemwithcomment{Spanish}{basic proficiency.}{}

\section{Technical skills}
\cvitem{Operating systems}{Good knowledge of GNU/Linux and Microsoft Windows environments.}
\cvitem{Programming}{I currently use FORTRAN77 (good knowledge) and Python (basic knowledge) for scientific computing. I'm familiar with the libraries: LAPACK, BLAS, Numpy, Matplotlib.}
\cvitem{Software}{I use Grace and Gnuplot for data visualization, LaTeX for scientific typing.}
% \cvdoubleitem{category 1}{XXX, YYY, ZZZ}{category 4}{XXX, YYY, ZZZ}
% \cvdoubleitem{category 2}{XXX, YYY, ZZZ}{category 5}{XXX, YYY, ZZZ}
% \cvdoubleitem{category 3}{XXX, YYY, ZZZ}{category 6}{XXX, YYY, ZZZ}

%\section{Interessi}
%\cvitem{Hobby}{Chitarrista amatoriale}
% \cvitem{hobby 2}{Description}
% \cvitem{hobby 3}{Description}

% \section{Extra 1}
% \cvlistitem{Item 1}
% \cvlistitem{Item 2}
% \cvlistitem{Item 3. This item is particularly long and therefore normally spans over several lines. Did you notice the indentation when the line wraps?}

% \section{Extra 2}
% \cvlistdoubleitem{Item 1}{Item 4}
% \cvlistdoubleitem{Item 2}{Item 5\cite{book1}}
% \cvlistdoubleitem{Item 3}{Item 6. Like item 3 in the single column list before, this item is particularly long to wrap over several lines.}

% \section{References}
% \begin{cvcolumns}
%   \cvcolumn{Category 1}{\begin{itemize}\item Person 1\item Person 2\item Person 3\end{itemize}}
%   \cvcolumn{Category 2}{Amongst others:\begin{itemize}\item Person 1, and\item Person 2\end{itemize}(more upon request)}
%   \cvcolumn[0.5]{All the rest \& some more}{\textit{That} person, and \textbf{those} also (all available upon request).}
% \end{cvcolumns}

% Publications from a BibTeX file without multibib
%  for numerical labels: \renewcommand{\bibliographyitemlabel}{\@biblabel{\arabic{enumiv}}}% CONSIDER MERGING WITH PREAMBLE PART
%  to redefine the heading string ("Publications"): \renewcommand{\refname}{Articles}

% \newpage

\nocite{*}
\bibliographystyle{unsrt}
\bibliography{publications}                        % 'publications' is the name of a BibTeX file

% Publications from a BibTeX file using the multibib package
%\section{Publications}
%\nocitebook{book1,book2}
%\bibliographystylebook{plain}
%\bibliographybook{publications}                   % 'publications' is the name of a BibTeX file
%\nocitemisc{misc1,misc2,misc3}
%\bibliographystylemisc{plain}
%\bibliographymisc{publications}                   % 'publications' is the name of a BibTeX file

% \clearpage

\newpage

\section{Scientific communications}

\cventry{September 4\textsuperscript{th}, 2020}{Talk}{}{titled ``Disorder-Enhanced and Disorder-Independent Transport with long range hopping: application to molecular chains in optical cavities'' at the conference ``CMD2020GEFES'', online}{}{}
\cventry{November 7\textsuperscript{th}, 2019}{Talk}{}{titled ``Interplay of cooperativity and functionality: from light-harvesting nanotubes to efficient photon-sensors'' at the conference ``Non-Hermitian Quantum Systems'', at Centro Internacional de Ciencias (Cuernavaca, Morelos, Mexico)}{}{}
\cventry{October 29\textsuperscript{th}, 2019}{Poster}{}{titled ``Efficient photo-detection and light harvesting via engineered cooperative effects'' at the conference ``Quantum Effects in Biological Systems (QuEBS)'', at Benem\'erita Universit\'ad Aut\'onoma de Puebla (Puebla, Mexico)}{}{}
\cventry{October 29\textsuperscript{th}, 2019}{Talk}{}{titled ``Macroscopic coherence as an emergent property in molecular nanotubes'' at the conference ``Quantum Effects in Biological Systems (QuEBS)'', at Benem\'erita Universit\'ad Aut\'onoma de Puebla (Puebla, Mexico)}{}{}
\cventry{October 23\textsuperscript{rd}, 2018}{Talk}{}{titled ``Non-Hermitian Hamiltonian approach to quantum transport in disordered networks with sinks: Validity and effectiveness'' at the conference ``Quantum Biology'', at Centro Internacional de Ciencias (Cuernavaca, Morelos, Mexico)}{}{}
\cventry{June 12\textsuperscript{th}, 2018}{Poster}{}{titled ``Temperature of a single chaotic eigenstate'' at the conference ``Chaos, quantum chaos and more'', at Centro Internacional de Ciencias (Cuernavaca, Morelos, Mexico)}{}{}
\cventry{March 27\textsuperscript{th}, 2018}{Talk}{}{titled ``Superabsorption of light: from Dicke to quantum engineering'' at Facoltà di Scienze Matematiche, Fisiche e Naturali - Università Cattolica del Sacro Cuore (Brescia, Italy)}{}{}
\cventry{December 12\textsuperscript{th}, 2017}{Talk}{}{titled ``Cooperative effects in light-harvesting systems'' at Facoltà di Scienze Matematiche, Fisiche e Naturali - Università Cattolica del Sacro Cuore (Brescia, Italy)}{}{}
\cventry{September 27\textsuperscript{th}, 2017}{Poster}{}{titled ``Cooperativity and scalability of light-harvesting devices by separating absorption from transmission'' at the conference ``Transport at the Nanoscale: Molecules, Graphene and more'', at Centro Internacional de Ciencias (Cuernavaca, Morelos, Mexico)}{}{}
\cventry{September 21\textsuperscript{th}, 2017}{Talk}{}{titled ``Cooperativity and scalability of light-harvesting devices by separating absorption from transmission'' at the conference ``Transport at the Nanoscale: Molecules, Graphene and more'', at Centro Internacional de Ciencias (Cuernavaca, Morelos, Mexico)}{}{}
\cventry{September 12\textsuperscript{th}, 2017}{Talk}{}{titled ``Cooperativity and scalability of light-harvesting devices by separating absorption from transmission'' at Instituto de F\'isica, Benem\'erita Universit\'ad Aut\'onoma de Puebla (Puebla, Mexico)}{}{}
\cventry{June 29\textsuperscript{th}, 2017}{Poster}{}{titled ``Decoupling absorption from transmission in light-harvesting devices'' at the conference ``XXII National Conference on Statistical Physics and Complex Systems'', at Università degli Studi di Parma (Parma, Italy)}{}{}

\newpage

\section{Attended Scientific Workshops, Schools and Courses}
\cventry{September 2\textsuperscript{nd} - 4\textsuperscript{th}, 2020}{Workshop}{}{CMD2020GEFES, online, organized by European Physical Society}{}{}
\cventry{November 4\textsuperscript{th} - 8\textsuperscript{th}, 2019}{Workshop}{}{Non-Hermitian Quantum Systems, at Centro Internacional de Ciencias (Cuernavaca, Morelos, Mexico)}{}{}
\cventry{October 27\textsuperscript{th} - 31\textsuperscript{st}, 2019}{Workshop}{}{Quantum Effects in Biological Systems (QuEBS), at Benem\'erita Universit\'ad Aut\'onoma de Puebla (Puebla, Mexico)}{}{}
\cventry{October 22\textsuperscript{nd} - 26\textsuperscript{th}, 2018}{Workshop}{}{Quantum Biology, at Centro Internacional de Ciencias (Cuernavaca, Morelos, Mexico)}{}{}
\cventry{June 4\textsuperscript{th} - 22\textsuperscript{nd}, 2018}{Workshop}{}{Chaos, quantum chaos and more, at Centro Internacional de Ciencias (Cuernavaca, Morelos, Mexico)}{}{}
\cventry{February 13\textsuperscript{th}, 2018}{PhD Course}{}{Materials and technologies for high-efficiency solar cells: from standards to nanostructures. Course held by By Prof. Antonio Terrasi (from Universit\`a degli Studi di Catania, Catania, Italy) at Facoltà di Scienze Matematiche, Fisiche e Naturali - Università Cattolica del Sacro Cuore (Brescia, Italy)}{}{}
\cventry{February 5\textsuperscript{th} - 8\textsuperscript{th}, 2018}{PhD Course}{}{Methods of numerical resolution of ODE systems: theory, implementation and applications. Course held by Prof. Adolfo Avella (from Universit\`a degli Studi di Salerno, Salerno, Italy) at Facoltà di Scienze Matematiche, Fisiche e Naturali - Università Cattolica del Sacro Cuore (Brescia, Italy)}{}{}
\cventry{January 11\textsuperscript{th} - 12\textsuperscript{th}, 2018}{PhD Course}{}{Understanding materials by molecular dynamics simulations. Course held by Claudia Caddeo, PhD (from IOM-CNR, Cagliari, Italy) at Facoltà di Scienze Matematiche, Fisiche e Naturali - Università Cattolica del Sacro Cuore (Brescia, Italy)}{}{}
\cventry{September 18\textsuperscript{th} - October 7\textsuperscript{th}, 2017}{Workshop}{}{Transport at the Nanoscale: Molecules, Graphene and more, at Centro Internacional de Ciencias (Cuernavaca, Morelos, Mexico)}{}{}
\cventry{August 7\textsuperscript{th} - October 30\textsuperscript{th}, 2017}{PhD Course}{}{Introduction to Classical and Quantum Chaos. Course held by Prof. Felix M. Izrailev at Instituto de F\'isica, Benem\'erita Universit\'ad Aut\'onoma de Puebla (Puebla, Mexico)}{}{}
\cventry{June 28\textsuperscript{th} - 30\textsuperscript{th}, 2017}{Workshop}{}{XXII National Conference on Statistical Physics and Complex Systems at Università degli Studi di Parma (Parma, Italy)}{}{}
\cventry{June 8\textsuperscript{th} - 22\textsuperscript{nd}, 2017}{PhD Course}{}{Wave processes in random media: physical principles, mathematical methods, and applications. Course held by Prof. Valentin Freilikher (from Bar-Ilan University Ramat-Gan, Israel) at Facoltà di Scienze Matematiche, Fisiche e Naturali - Università Cattolica del Sacro Cuore (Brescia, Italy)}{}{}
\cventry{June 27\textsuperscript{th} - July 1\textsuperscript{st}, 2016}{Workshop}{}{IWDS10 - International Workshop on Disordered Systems, at Facoltà di Scienze Matematiche, Fisiche e Naturali - Università Cattolica del Sacro Cuore (Brescia, Italy)}{}{}
\cventry{June 20\textsuperscript{th} - 24\textsuperscript{th}, 2016}{School}{}{2nd School on Scientific Data Analytics and Visualization, at CINECA (Bologna, Italy)}{}{}

% -----       letter       ---------------------------------------------------------
% % recipient data
% \recipient{Company Recruitment team}{Company, Inc.\\123 somestreet\\some city}
% \date{January 01, 1984}
% \opening{Dear Sir or Madam,}
% \closing{Yours faithfully,}
% \enclosure[Attached]{curriculum vit\ae{}}          % use an optional argument to use a string other than "Enclosure", or redefine \enclname
% \makelettertitle

% Lorem ipsum dolor sit amet, consectetur adipiscing elit. Duis ullamcorper neque sit amet lectus facilis3is sed luctus nisl iaculis. Vivamus at neque arcu, sed tempor quam. Curabitur pharetra tincidunt tincidunt. Morbi volutpat feugiat mauris, quis tempor neque vehicula volutpat. Duis tristique justo vel massa fermentum accumsan. Mauris ante elit, feugiat vestibulum tempor eget, eleifend ac ipsum. Donec scelerisque lobortis ipsum eu vestibulum. Pellentesque vel massa at felis accumsan rhoncus.

% Suspendisse commodo, massa eu congue tincidunt, elit mauris pellentesque orci, cursus tempor odio nisl euismod augue. Aliquam adipiscing nibh ut odio sodales et pulvinar tortor laoreet. Mauris a accumsan ligula. Class aptent taciti sociosqu ad litora torquent per conubia nostra, per inceptos himenaeos. Suspendisse vulputate sem vehicula ipsum varius nec tempus dui dapibus. Phasellus et est urna, ut auctor erat. Sed tincidunt odio id odio aliquam mattis. Donec sapien nulla, feugiat eget adipiscing sit amet, lacinia ut dolor. Phasellus tincidunt, leo a fringilla consectetur, felis diam aliquam urna, vitae aliquet lectus orci nec velit. Vivamus dapibus varius blandit.

% Duis sit amet magna ante, at sodales diam. Aenean consectetur porta risus et sagittis. Ut interdum, enim varius pellentesque tincidunt, magna libero sodales tortor, ut fermentum nunc metus a ante. Vivamus odio leo, tincidunt eu luctus ut, sollicitudin sit amet metus. Nunc sed orci lectus. Ut sodales magna sed velit volutpat sit amet pulvinar diam venenatis.

% Albert Einstein discovered that $e=mc^2$ in 1905.

% \[ e=\lim_{n \to \infty} \left(1+\frac{1}{n}\right)^n \]

% \makeletterclosing

%\clearpage\end{CJK*}                              % if you are typesetting your resume in Chinese using CJK; the \clearpage is required for fancyhdr to work correctly with CJK, though it kills the page numbering by making \lastpage undefined
\end{document}


%% end of file `template.tex'.
